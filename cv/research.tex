%-------------------------------------------------------------------------------
%	SECTION TITLE
%-------------------------------------------------------------------------------
\cvsection{Research experience}


%-------------------------------------------------------------------------------
%	CONTENT
%-------------------------------------------------------------------------------
\begin{cventries}

%---------------------------------------------------------
  \cventrySimple
    {Research assistant} % Job title
    {Polytechnic University of Turin} % Organization
    {Turin, Italy} % Location
    {Dec. 2021 - Today} % Date(s)
  \cvtext{
    I am conducting research on bioinformatics algorithms and technologies, and on heterogeneous computing.
  I am currently working on three different topics:
    \begin{cvitems2}
      \item \textbf{RNA structural alignment.} Improvement of the C++ program LaRA2 for pairwise RNA structural alignment, by
      reducing the required time and memory footprint, especially for long sequences. The final objective is to
      use LaRA2 in conjunction with other tools to perform multiple structural sequence alignment and folding of COVID sequences, in a
    reasonable time.
      \item \textbf{Somatic structural variations (SVs) discovery.} Development of a pipeline for the discovery of somatic mutations
    in ONT long read samples, coming from an oncology case study. The dataset contains paired samples of healthy and cancerous cells.
    The basic idea is to discover SVs in both samples (using tools as CuteSV), to then identify the somatic SVs by subtracting the
    germline mutations from the SVs in the cancerous sample. The parsing and subtraction of SVs is being developed in Python.
      \item \textbf{Benchmarking of Intel's Lava framework.} Benchmarking of the recently released Lava framework by Intel,
    for heterogeneous computing including neuromorphic hardware (like Intel's Loihi and Loihi2 chips). The benchmark is being
    developed primarily in Python (as Lava is written in Python as well), and takes some inspiration from MPI benchmarks.
    \end{cvitems2}
  }
%---------------------------------------------------------
\end{cventries}