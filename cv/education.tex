%-------------------------------------------------------------------------------
%	SECTION TITLE
%-------------------------------------------------------------------------------
\cvsection{Education}


%-------------------------------------------------------------------------------
%	CONTENT
%-------------------------------------------------------------------------------
\begin{cventries}

%---------------------------------------------------------
  \cventry
    {B.S. in Electronic Engineering} % Degree
    {University of Antioquia} % Institution
    {Medellin, Colombia} % Location
    {Mar. 2010 - Aug. 2015} % Date(s)
    {
      \begin{cvitems} % Description(s) bullet points
        \item {\textbf{Grade:} 4.5 / 5}
        \item {\textbf{Achievements:} Honor Student in the 1st, 2nd and 4th semesters.}
      \end{cvitems}
    }

  \cventry
    {M.Sc. in Computer Engineering}
    {Polytechnic University of Turin}
    {Turin, Italy}
    {Sept. 2015 - Jul. 2018}
    {
      \begin{cvitems}
        \item {\textbf{Pathway:} Embedded systems}
        \item {\textbf{Grade:} 110 / 110 cum laude}
        \item {\textbf{Thesis title:}  A Secure Password Wallet based on the
        SEcube framework}
        \item {\textbf{Thesis summary:} It is very common nowadays to rely on software
        applications such as web browsers for the management and storage of personal
        passwords.
        In my work I presented and alternative hardware solution based on the SEcube™
        (Secure Environment cube) framework, which consist of an open source
        security-oriented hardware platform and a set of open source software libraries.
        The developed application, named SEcubeWallet,
        was written in C/C++ and Qt. It manages passwords using secureSQlite,
        one of the SEcube™ libraries, which wraps the functionalities of the SQlite
        standard to create secured databases. The passwords are encrypted using a
        personal SEcube™ device (which looks like a regular USB pendrive),
        and can only be decrypted if the device is connected to the PC and the
        user authenticates using a master password.
        As the cryptographic operations are performed by the device, not by the host PC, the
        passwords can be accessed in any computer where an appropriate
        version of Qt is installed and the device is connected.
        The application is cross-platform and the GUI is easily configurable.
        The user can easily create, delete, open,
        and modify password wallets. Additionally, the application can suggest strong Passwords and
        Passphrases and verify the entropy of the ones provided by the user.
        }
      \end{cvitems}
    }

%---------------------------------------------------------
\end{cventries}
